%%%
%%%  Template to prepare a defense of Bc./Mgr./... thesis
%%%  to be presented at MFF
%%%  (unofficial)
%%%
%%%  AUTHOR:  Arnošt Komárek
%%%           Department of Probability and Mathematical Statistics
%%%           Faculty of Mathematics and Physics, Charles University in Prague
%%%
%%%  LOG:    20150505  created by modification of some previous personal presentations
%%%          20170522  update related to new MFF logo
%%%  
%%%  ===========================================================================
\documentclass[c, 10pt]{beamer}


%%%%% Package needed if some accented letters in the presentation
%%%%% -------------------------------------------------------------
\usepackage[utf8]{inputenc}


%%%%% Most beamer settings and other LaTeX commands
%%%%% are provided in the file below
%%%%% -----------------------------------------------------
\input{MFF_Present.sty}


%%%%% \ifCZversion is defined inside MFF_Present.sty
%%%%% to distinguish between Czech and English presentations
%%%%% ------------------------------------------------------
\CZversiontrue       %% for presentations in Czech (Slovak)
%\CZversionfalse      %% for presentations in English


%%%%% Uncomment appropriate choice below if you wish to create
%%%%% notes for audience having 2 or 4 slides on each (A4) page.
%%%%% -------------------------------------------------------------
\usepackage{pgfpages}
%\pgfpagesuselayout{4 on 1}[a4paper, landscape, border shrink=5mm]
%\pgfpagesuselayout{2 on 1}[a4paper, border shrink=5mm]

\graphicspath{ {FigureLayout/} }  % path to images folder

\usepackage{dcolumn}        % improved alignment of table columns
\usepackage{booktabs}       % improved horizontal lines in tables


%%%%% Basic settings of the document
%%%%% (will be automatically used to create a title page, foots etc.)
%%%%% --------------------------------------------------------------------

  %%% Main title
  %%% - short and long version
  %%%   --> will appear on place where \inserttitle and \insertshorttitle commands used
  %%%   --> if the full title is short enough, both short and long versions might be the same
\title[Board game with artificial intelligence]{%                       
       Board game with artificial intelligence}

  %%% Subtitle (comment it if you do not want to have it)
  %%%   --> will appear on place where \insertsubtitle and \insertshortsubtitle commands used
\subtitle[]{Obhajoba bakalářské práce}

  %%% Author
  %%% - as "short" version, link to the author's webpage is used
  %%%   (e.g., e-mail is also a useful alternative)
  %%%   --> will appear on places where \insertauthor and \insertshortauthor commands used
\author[daniel.crha(at)gmail.com]{%
        Daniel Crha}

  %%% Author's affiliation
  %%% - can be fully commented for defense presentation
  %%%   --> will appear on places where \insertinstitute and \insertshortinstitute commands used
\institute[KTIML]{%
           Katedra teoretické informatiky a~matematické logiky}

  %%% Date of presentation
  %%% - replace it by real date in case of a defense presentation
  %%%   --> will appear on places where \insertdate and \insertshortdate commands used
\date[7.7.2020]{%
      7. července 2020}


\begin{document}

%%%%% Title slide
%%%%% =====================================================================================
\frame[plain]{\titlepage}


%%%%% Fictitious introductory section
%%%%% =====================================================================================
\section{Úvod}
%\framesection{}     %%% Uncomment it to get a special slide with the section title
                     %%% - not really needed for a presentation lasting 10 minutes


  %%%%% Slide
  % ----------------------------------------------------------------------------------------
\begin{frame}\frametitle{Úvod}
\framesubtitle{Problematika}

    \begin{itemize}\itemsep=1em
    \item Klasické problémy teorie her jsou dobře prozkoumané
        \begin{itemize}\color{colTwo}\itemsep=1ex
            \item Mají i dobrou podporu
        \end{itemize}
    \item Tyto problémy ale často nemodelují reálný svět
    \end{itemize}
\end{frame}


  %%%%% Slide
  % ----------------------------------------------------------------------------------------
\begin{frame}\frametitle{Úvod}
    \framesubtitle{Složitější vlastnosti v teorii her}

    \begin{itemize}\itemsep=1em
    \item Chtěli bychom zkoumat vlastnosti, které reálný svět modelují
    \item Konkrétně:
        \begin{itemize}\color{colTwo}\itemsep=1ex
            \item Neúplnost informace
            \item Více hráčů
            \item Prvky náhody
            \item Netriviální větvící faktor
        \end{itemize}
    \item Jedná se o méně zkoumanou oblast
    \end{itemize}
\end{frame}

  %%%%% Slide
  % ----------------------------------------------------------------------------------------
\begin{frame}\frametitle{Cíle}
\framesubtitle{}

    \begin{itemize}\itemsep=1em
        \item Navrhnout hru se zmíněnými vlastnostmi
        \item Hru implementovat s~podporou pro umělou inteligenci
        \item Implementovat a~porovnat několik inteligencí
    \end{itemize}
\end{frame}

  %%%%% Slide
  % ----------------------------------------------------------------------------------------
\begin{frame}\frametitle{Návrh hry}
\framesubtitle{Pravidla a vlastnosti}
    
    \begin{itemize}\itemsep=1em
        \item Hra se jmenuje \alert{Colonizers}
        \item 4 hráči
        \item Hra se hraje v kolech s~fázemi
    \end{itemize}
\end{frame}

  %%%%% Slide
  % ----------------------------------------------------------------------------------------
\begin{frame}\frametitle{Návrh hry}
\framesubtitle{Pravidla a vlastnosti}
    \begin{figure}[ht]
        \centerline{\mbox{\includegraphics[width=110mm]{player}}}
        \caption{Přehled hráče.}\label{ud:player}
    \end{figure}
\end{frame}

  %%%%% Slide
  % ----------------------------------------------------------------------------------------
\begin{frame}\frametitle{Implementace}
\framesubtitle{Technologie}
    \begin{itemize}\itemsep=1em
        \item GUI - Electron + Angular
        \item Logika - C\# (.NET Core)
            \begin{itemize}\color{colTwo}\itemsep=1ex
                \item ASP.NET Core jako backend pro GUI
            \end{itemize}
        \item Umělé inteligence - Python
    \end{itemize}
\end{frame}

  %%%%% Slide
  % ----------------------------------------------------------------------------------------
\begin{frame}\frametitle{Implementace}
\framesubtitle{Rozhraní pro umělou inteligenci}
    \begin{itemize}\itemsep=1ex
        \item Inteligence implementuje bázovou třídu
        \begin{itemize}\color{colTwo}\itemsep=1ex
            \item Na ní implementuje callback
        \end{itemize}
        \item Bázová třída má další funkcionality
        \begin{itemize}\color{colTwo}\itemsep=1ex
            \item Determinizace
            \item Simulace
        \end{itemize}
        \item Soubor s umělou inteligencí je spouštěn jako \alert{\_\_main\_\_}
    \end{itemize}
\end{frame}

  %%%%% Slide
  % ----------------------------------------------------------------------------------------
\begin{frame}\frametitle{Umělé inteligence}
\framesubtitle{Implementované typy}
    \begin{itemize}\itemsep=1ex
        \item Náhodná inteligence
        \item Heuristická inteligence
        \item MaxN
            \begin{itemize}\color{colTwo}\itemsep=1ex
                \item Rozšíření Minimaxu na hry s více hráči
                \item Determinizace + poziční vyhodnocování
            \end{itemize}
        \item ISMCTS
            \begin{itemize}\color{colTwo}\itemsep=1ex
                \item Monte Carlo metoda
                \item Determinizace + simulace
            \end{itemize}
    \end{itemize}
\end{frame}

  %%%%% Slide
  % ----------------------------------------------------------------------------------------
\begin{frame}\frametitle{Experimenty}
\framesubtitle{Vybalancování hry}
    \begin{table}[h!]
        \centering
        \begin{tabular}{l@{\hspace{1.5cm}} c c c c}
        \textbf{Pozice} & \textbf{1} & \textbf{2} & \textbf{3} & \textbf{4} \\
        \midrule
        Výhry            & 230 & 202   & 282   & 286 \\
        Prohry          & 415 & 298   & 152   & 135 \\
        Průměrný výsledek    & 2.8 & 2.67 & 2.302 & 2.228 \\
        \bottomrule
        \end{tabular}
        \caption{Výsledky hraní heuristických inteligencí, 1000 her.}\label{tabex:heuristicwins}
    \end{table}
\end{frame}

  %%%%% Slide
  % ----------------------------------------------------------------------------------------
\begin{frame}\frametitle{Experimenty}
\framesubtitle{Vybalancování hry}
    \begin{itemize}
        \item $\chi^{2}$ test pro binomické rozdělení
        \item Odchylka proher na 1. a 4. místě je statisticky významná
    \end{itemize}
\end{frame}

  %%%%% Slide
  % ----------------------------------------------------------------------------------------
\begin{frame}\frametitle{Experimenty}
\framesubtitle{Porovnání inteligencí}
    \begin{table}[h!]
        \centering
        \begin{tabular}{l@{\hspace{1.5cm}} c c c c}
        \textbf{AI} & \textbf{Random} & \textbf{Heuristic} & \textbf{MaxN} & \textbf{ISMCTS} \\
        \midrule
        Výhry            & 0   & 5     & 8     & 37 \\
        Prohry          & 40  & 1     & 9     & 0 \\
        Průměrný výsledek    & 3.8 & 2.38  & 2.54  & 1.28 \\
        \bottomrule
        \end{tabular}
        \caption{Výsledky hraní všech čtyř inteligencí dohromady, 50 her.}\label{tabex:oneofeach}
    \end{table}
\end{frame}

  %%%%% Slide
  % ----------------------------------------------------------------------------------------
\begin{frame}\frametitle{Experimenty}
\framesubtitle{Porovnání inteligencí}
    \begin{table}[h!]
        \centering
        \begin{tabular}{l@{\hspace{1.5cm}} c c c c}
        \textbf{AI} & \textbf{Heuristic} & \textbf{MaxN} \\
        \midrule
        Výhry            & 35   & 15   \\
        Prohry          & 10   & 40   \\
        Průměrný výsledek    & 2.13 & 2.87 \\
        \bottomrule
        \end{tabular}
        \caption{Výsledky hraní dvou heuristických a dvou MaxN inteligencí, 50 her.}\label{tabex:heurmaxn}
    \end{table}
\end{frame}

%%%%% Fictitious final section
%%%%% ===================================================================================
\section{Závěr}
%\framesection{Závěr}     %%% Uncomment it to get a special slide with the section title
                          %%% - not really needed for a presentation lasting 10 minutes

  %%%%% Slide
  % ----------------------------------------------------------------------------------------
\begin{frame}[plain]         %%% plain = no title etc.

%%% Rámeček
\begin{beamercolorbox}[center, sep=2pt, rounded=true, shadow=true]{mffboxcol}
\LARGE\bfseries \alert{Děkuji za pozornost!}
\end{beamercolorbox}

\vspace{5em}
\begin{beamercolorbox}[center, sep=2pt, rounded=true, shadow=true]{mffboxcol}
Za ochotu a~čas mně věnovaný při přípravě této bakalářské práce děkuji též svému vedoucímu \alert{Mgr. Martinu Pilátovi, Ph.D.}
\end{beamercolorbox}
\end{frame}


%%%%% Fictitious reactions to comments of the reviewer
%%%%% ===================================================================================
\section{Reakce na připomínky oponenta}
%\framesection{}

  %%%%% Slide
  % ----------------------------------------------------------------------------------------
\begin{frame}\frametitle{Připomínky oponenta}
\framesubtitle{Determinizace}
    \begin{itemize}\itemsep=1ex
        \item Engine hry sleduje množiny informací pro jednotlivé hráče
        \item Konkrétně se determinizují:
            \begin{itemize}\color{colTwo}\itemsep=1ex
                \item Colonist pro ostatní hráče
                \item Moduly v rukou ostatních hráčů
            \end{itemize}
    \end{itemize}
\end{frame}

  %%%%% Slide
  % ----------------------------------------------------------------------------------------
\begin{frame}\frametitle{Připomínky oponenta}
\framesubtitle{Běhové časy strategií}
    \begin{table}[h!]
        \centering
        \begin{tabular}{l@{\hspace{1.5cm}} c c c c}
        \textbf{AI} & \textbf{Random} & \textbf{Heuristic} & \textbf{MaxN} & \textbf{ISMCTS} \\
        \midrule
        Běhový čas            & 1ms & 1ms & 24s & 51s \\
        \bottomrule
        \end{tabular}
        \caption{Průměrný běhový čas potřebný pro jedno rozhodnutí.\linebreak10 her, zaokrouhleno.}\label{tabex:oneofeach}
    \end{table}
\end{frame}

\end{document}
